\documentclass[letter:wpaper]{article}
\usepackage{isea}
% UPDATE: See the local /.vscode/settings.json
% The pdftex option is necessary for the pdfinfo command to work, for the graphicx package to work, and for the correct fonts to work. It requires the pdfLaTeX renderer
% [1] Select the Preferences: Open User Settings (JSON) command in the Command Palette (⇧⌘P)
% [2] Find the latex-workshop.latex.recipes setting and move the "xelatexmk" recipe to the top of the list. This will make xelatexmk the default recipe for building LaTeX files.
%
%   "latex-workshop.latex.recipes": [
%     {
%       "name": "latexmk",
%       "tools": ["latexmk"]
%     },
%     {
%       "name": "latexmk (xelatex)",
%       "tools": ["xelatexmk"]
%     }
% [3] There is no 3. There is already a "latex-workshop.latex.tools" setting for "latexmk" that uses the -pdf flag and so will use pdfLaTeX. I suppose [3] is for putting xelatexmk back as the default recipe
\usepackage[pdftex]{graphicx}
\pdfinfo{
    /Title (Art as Technology)
    /Author (Kynan Stewart Hughes)
}
\usepackage{times}
\usepackage{helvet}
\usepackage{courier}
\usepackage[numbers]{natbib}
% The file isea.sty is the style file for ISEA 2025 proceedings.
%
\title{Art as Technology (v1.1.0.1)}
\author{Kynan Stewart Hughes\\
Creativity and Cognition Studios\\
University of Technology Sydney\\
Sydney, Australia\\
kynan.s.hughes@student.uts.edu.au\\
\newline
\newline
}
\setcounter{secnumdepth}{0}

\begin{document} 
% Setting the default tolerance level
% \tolerance=200
% Setting a high tolerance level
% \tolerance=10000
% Some more drastic measures for hbox issues
% \emergencystretch=4em
% \raggedright
\maketitle
\begin{abstract}

    This paper argues for the conceptualisation of art as a technology, exploring its implications through the lens of complexity. It challenges the traditional partitioning of art and technology, proposing that art is a codifying system that dislocates objects and materials from their original functions and meanings, and encodes them with new affective potential. In other words, it is a kind of technology that programs our capacity to attribute affective potential to objects. The analysis suggests that rethinking art as a form of technology has the potential to expand the creative possibilities for artists working with emerging technologies and offers a pathway to reconsidering our relationship with our technologies. The conclusion invites artists who work with emerging technologies to explore an ethical mode of crafting that foregrounds interdependence and complexity.

\end{abstract}

\keywords{Keywords}

Art, technology, complexity, aesthetics, affect, information, craft, systems, emergence, meaning, constraints, coarse–graining

\section{Introduction}

    This paper explores an idea occasionally \citep[pp.74–75]{SauvagnarguesArtmchns2016} \citep{GellThTchnlgyOfEnchntmnt1992} and fleetingly \citep[p.202]{ZepkeOSullivanDlzCntmprryArt2010} articulated – that art is a technology. It is an idea that grounds art as a material practice, unique and special as a technology, not different from technology. As a bonus, the same move makes technology make a different kind of sense as a field of practice that includes artmaking, re-opening, among other things probably, an ethical dimension to technology that is otherwise too easily ignored. 
    
\section{A Definition of Technology} 

    According to Brian Arthur a technical object is always ``a phenomenon captured and put to use.'' \citep[p.53]{theNatureOfTechnology2009}. To put it another way, technology is ``programming [...] phenomena for [...] purpose.'' \citep[p.53]{theNatureOfTechnology2009}. This distillation holds for all technologies, contemporary and historical. ``Phenomena'' don't have to be physical, like fire or electricity. They may be ``behavioural or organisational  effects \citep[p.55]{theNatureOfTechnology2009}, or ``truism[s] of nature'' \citep[p.45]{theNatureOfTechnology2009}.

    \subsection{Example: Money}

    For example, Money is a technology.

    \begin{quote}
        The monetary system makes use of the ``phenomenon'' that we trust a medium has value as long as we believe that others trust it has value and we believe this trust will continue in the future. \citep[p.55]{theNatureOfTechnology2009}
    \end{quote}

\section{A Definition of Art}

    Jacques Rancière has called contemporary Western art the \emph{aesthetic regime}. It is not a single coherent paradigm, but a plurality of ``frequently different, and sometimes contradictory, ways of thinking'' \citep[p.8]{RanciereMdrnTms2022} organised around the idea of aesthetic experience. It emerged at the end of the eighteenth century and is strongly identified with Emanuel Kant's \emph{Critique of Judgement} \citep[pp.23–24]{RancierPltcsOfThAsthtcs2004}. 

    Within this regime, all distinctions and differences operate in relation with the idea of aesthetic experience. Styles like Modern and Post–Modern, for example, are simply different strategies for taking meaningful action within the aesthetic regime \citep[p213]{ZepkeSblmArt2017}. Art which is anti–aesthetic is operating within the aesthetic regime by being self–consciously critical of it\footnote{
        Reference.
    }. Conceptual art is art that invites us to consider the aesthetic potential of ideas\footnote{
        Reference.
    }.

    Art is a complex codifying system that dislocates objects from their original functions and meanings and gives them new functions and meanings. The ``dissensual operation'' of creating an artwork, for example by working a material into a form, or by setting the state of pixels on a screen, or by appropriating a ready–made object, literally ``transforms a given form or body into a new one.'' \citep[p.54]{RancierThEmncptdSpcttr2009}.

\section{Affect}

    The concept of \emph{affect}, which comes from Deleuze and Guattari via Brian Massumi \citep{MassumiTheAtnmyOfAffct1995}, is a way of thinking about how things affect other things in the complex interconnected systems of material reality. Affect is a kind of intensive, interactive event. It is human and inhuman, organic and inorganic.

    Affect is when emergence happens. It is a quantum shift ``fed forward'' across progressively higher levels of reality \citep[p.37]{MassumiPrblsFrThVrtl2002} until it may register on human perception as apparent cause and effect. This apparent causality echoes back as ``downward causation'' \citep[p.?]{FlackCrsGrnng2017} that constrains and influences the behaviour of system components at all levels.
    
    Humans are particularly good at attributing affective regularity to objects \citep{FristonThFrEnrgPrncpl2010} \citep{DeaconTheSymbolicSpecies1998}. We notice patterns. The regular cycles of seasons and plants, the behaviours of animals. The organisation of sounds into music and language.
    
    Art, the aesthetic regime, is system of codes that exploits our capacity for noticing affective regularity. It is a hack on our capacity for attributing affective power to objects. Through a process of becoming art, an object is packed with affect, like a battery holding a charge. 
    
    The affective charge of an art object can consist of any combination of qualities – strong, weak, sensorial, semantic, abstract, pleasant and unpleasant et cetera. It can be a sense of the beautiful, or it can involve concepts and feelings that are more difficult to pin down. It can, like a Patricia Piccinini sculpture, strobe between incompatible affective states, such as the cute and the grotesque. Like a readymade, it can call attention to the codifying power of art itself.

    The relatively recent move to equate aesthetic experience with affect is a shift away from the idea that some normative experiences are better than others and that an artwork is a moral lesson, and towards the messy reality of material processes. It contextualises aesthetic experience in a complex ecology of affect. As Ben Highmore put it, referencing Guattari, art is now about “complex affective and intensive exchanges, situated in the broader ecology of the world.” \citep[p.155]{HighmoreBttrAftrTst2010}

    \section{Purpose}

    According to Felix Guattari, art ``confers a function of sense and alterity to a subset of the perceived world.'' \citep[p.131]{GuattariChsmss1995}. An artwork is a special kind of object that is \emph{about} something.
    
    Arthur Danto said that artworks are ``embodied meaning'' \citep[p.125]{DantoEmbdMnngs2007}, and meaning is a useful way to think about the affective purpose of art objects as long it is remembered that an art object's meaning is always contextual\footnote{
        Manuel DeLanda has pointed out that the words ``meaning'' in the expressions ``this word has a meaning'' and ``this life has a meaning'' are two entirely different words'' \citep[pp.40–41]{DeLandaCsltyAndMnng2018}. Suggesting, as for example Noël Carroll has, that Danto's idea of embodied meaning is too narrow, is to assume the first sense of the word meaning, which is the simple signifier/signified sense.
    }. Massumi has said, ``meaning is [...] a network of enveloped material processes'' and, quoting Deleuze and Guattari, ``‘A thing has as many meanings as there are forces capable of seizing it.’'' \citep[p.10]{MassumiAUsrsGdTCptlsmAndSchzphrn1992}.

    However, the question of purpose is not easily left at that because it has traditionally been used to differentiate artmaking from other crafts. For Kant, art objects, which for him meant beautiful objects, were necessarily ``purposive without a purpose'' \citep[p.57]{KantCrtqOfJdgmnt}. The very essence of an artwork was its lack of functional purpose (as if being art was not a functional purpose).
    
    This is a weirdly circular and weirdly enduring idea, and it has been the bases for other weird ideas such as the idea that art objects are ``autonomous''\footnote{
        Reference: Greenberg
    }. It only makes sense if one is already looking for a reason why art is different to other kinds of making and doing, which of course Kant was.
    
    It also depends on an assumption that objects are fixed, independent things, and therefore so are their purposes. Over the last eighty year or so, an awareness of the ubiquity of complexity has emerged in Western sciences and philosophies, and it is now possible to see that all objects, along with their purposes and meanings, are entirely dependant on context.

    For example, as well has having a embodied meaning, which might be termed their ‘art purpose’, art objects always also have other, ‘non–art’ purposes. While the art purpose of a painting may simply be to evoke an experience of beauty, at the same time it may confer status, be an investment, or start a revolution. If art objects have as many meanings as there are forces capable of capturing them \citep[p.4]{DeleuzeNtschAndPhlsphy2006}, then the same thing can be said of their non–art purposes.

    The affective charge of an art object may derive, to some extent, from a non–art purpose, but is never reducible to it. There is always ``contextual excess or remainder'' \citep[p.252]{MassumiPrblsFrThVrtl2002}.
    
    When art purpose and non–art purpose become confused in an artwork, it is the art purpose that suffers. This is particularly relevant for artists who work with emerging technologies because the non–art purposes of emerging technologies are exciting and not fully understood. The meaning of an art object that incorporates emerging technology is likely to be ambiguous because it is entangled with the non–art purposes.
    
	In the early 2000s, practitioners of locative art were criticised for their uncritical use of mapping and networked, location-aware, mobile technologies, as well as for their collaborations with industry. It was alleged that they lacked a structure of accountability and ethics, were ushering in a `society of control', and were turning the media-art conference circuit into a `shopping-driven [...] spectacle' \citep[p.358]{beyondLocativeMedia2006}. Locative art was described as a ``technocentric fantasy'' that ``downplays [...] history'' \citep[para. 2]{questioningTheFrame2004}.
    
    Clearly the art purpose of locative art was being interfered with by the artists' failure to account for the non–art purposes and potential inherent in applications of these technologies. Today, as we grapple with the conjunction of rampant technology-fuelled capitalism and mass surveillance, the criticisms seem to have been validated. 

    Incorporating non–art purpose into the meaning of an art object is a risky strategy, which is what makes it interesting. In socially connected art, participatory art, and art that is made to be used (craft and design), for example, non–art purpose and art purpose are may be productively and delightfully entangled. 
    
    However, should the meaning of an art object become reducible, for the viewer, to a non–art purpose, it would have ceased to be an art object, as the next example, drawn from my own experience, shows clearly.

    \subsubsection{Example: \emph{FlowAttractor}}

    \begin{figure}[h]
        \includegraphics[width=3.31in]{flow-attractor.png}
        \caption{\emph{FlowAttractor} models complex workflows as flying cubes. \copyright Respect Copyright.}
        \label{fig:flow-attractor}
    \end{figure}

    In \emph{FlowAttractor} (Figure~\ref{fig:flow-attractor}), the flow of the blocks represents the otherwise invisible flow of work items through the software production system of a large organisation. The piece is an experiment to find out what can happen if an art–like object is integrated into a business context. Its non-art purpose within the organisational context is to catalyse an awareness of the effects of making small decisions designed to improve the flow of work through the system. It works surprisingly well in this way because augmented reality models enable good reasoning with respect to complicated workflows. However, because its purpose is, in that context, ultimately reducible to its non–art purpose, it is effectively not art.
    
    An contemporary patch on Kant's ideas of art and purposelessness proposes that it is the very contextual, contingent nature of an art object's purpose that defines art as a unique practice. For example, Kant's formulation is reinterpreted by Jason Hoelscher to mean ``not that art lacks purpose, but that art's purpose is contextually complex and indeterminate, and so remains open to transformation, [...] many layered, and multidimensional.'' \citep[p.25]{HoelscherThPtcsOfPhsSpc2014}.

    This allows for the complex reality of art objects but denies the same complexity to technical objects. For this indeterminacy of purpose to be the factor that defines art as something different to technology, technical objects would have to have limited and defined purposes, and of course they don't. Viagra is an angina treatment. 

\section{A Fold in the Distribution of the Sensible}
    
    According to science, a phenomenon called ``coarse graining'' occurs when a subsystem with apparently emergent properties is treated as a single entity by components of a complex system for predictive purposes. Coarse graining works by ignoring the details. When it works, it is an efficient, ``lossy but true'' \citep[p.4]{FlackCrsGrnng2017} strategy.

    Sometimes, however, it is just lossy \citep[p.8]{FlackCrsGrnng2017}. People are hard–wired to look for pattern and it has been evolutionarily advantageous for us to see patterns, to the extent that we tend to imagine them \citep{FristonThFrEnrgPrncpl2010}.
    
    It seems likely that this has been the case with the perceived pattern of difference that separates art and technology. The split between the art and other kinds of making is, I suggest, a dodgy piece of coarse graining that has been with us for about six hundred years. This is possible because the aesthetic regime is overlaid upon an older regime that laid the foundation for this split between art and technology. 
    
    \subsubsection{The Regime of Representation}

    The idea of there being a qualitative difference between the practices we now call ‘the arts‘ and other ways of doing and making began taking shape during the Renaissance \citep[p.136]{TatarkiewiczWhtIsArt1971}. By the 17th century the classification of various practices as \emph{the arts} had become firmly established. Rancière has called this \emph{partitioning of the sensible} the ``regime of representation''\footnote{
        Reference.
    }.
    
    Although the various \emph{arts} existed – understood as forms of knowledge and their applications – they were not recognised as part of a singular, overarching category of human experience called ``art''. The arts – music, literature, sculpture, painting, et cetera – were disparate practices serving different social functions, and were situated within a stratified system that categorised both activities and the individuals engaged in producing them. The job of the arts was to represent the world as a unity of sensible order. A place for everything and everything in its place, including people.
    
    The aesthetic regime -- art as we know it today -- is a tactic that cuts across the regime of representation, but we still observe the fold of difference that separates the arts from other technical practices. For example, we organise our our university faculties and school curriculums along it: science, technology, engineering and maths on one side, the arts and humanities on the other. 
    
    It is this fold in the ``distribution of the sensible'' \citep[p.42]{RancierPltcsOfThAsthtcs2004} that the idea of art as a technology smooths out. What would it be like, one wonders, for this fold not to exist?  A look at the situation that proceeded the emergence of regime of representation reveals a regime in which the fold does not exist.
    
    \subsection{The Ethical Regime of Images}

    Heidegger, in his essay ``The question concerning technology'', pointed out that ``techne'' was the ancient Greeks' word for skill, craft, and technique. The term covered what we might now call ‘the arts’ as well as science, and technical domains like sword-making and shipbuilding \citep[p34]{HeideggerThQstnCncrngTchnlgy1954}.
    
    Rancière has called the mode of thought that prevailed at this time the \emph{ethical regime of images}, the word ``image'' referring to the idea that all made things in the world are instances of ideal, universal forms.

    At this time, ethical concerns were primary. The ``end or purpose'' of crafted objects mattered a great deal: the uses they were put to, the effects they resulted in -- in general the way their modes of being affected the ``ethos'', ``the mode of being of individuals and communities'' \citep[pp.20–21]{RancierPltcsOfThAsthtcs2004}.

    Plato thought in terms of \emph{true arts}, which are forms of knowledge based on the imitation of a model with precise ends, and \emph{lesser arts} that simply imitate appearances \citep[p.20]{RancierPltcsOfThAsthtcs2004}. These ideas, which now seem quaint, were nonetheless an ethical framework informed by a concern for the connection between actions and their potential effects.

    Like the regime of representation, the ethical regime of images still operates as a kind of substrate. We care about who produces what and for what purpose, especially when it comes to art.
    
    In Australia in 2021, an art festival in Hobart planned to include a piece by Santiago Sierra, which involved a call for donations of blood from descendants of First Nations peoples who survived the genocidal effects Tasmanian colonisation. First nations people around Australia argued that the piece would emphasise the bloody aspects of colonization for no positive effect. Rappers Tasman Keith and Briggs commented on Instagram that they ``already gave enough blood" \citep{DrkMfBld2021}. The festival organisers apologised and cancelled the piece.
    
    Somewhere along the line technological crafting became decoupled, in a way that artmaking didn't, from the kind of exquisite sensitivity to affect which is ethics. Technology evolves, it seems, regardless of how we feel about it. For example, while many people fear the potential effects of AI generated content, while we might fear losing our jobs or our signature styles to the proliferation of technological systems that can do what we do at a different economic scale, our feelings on these matters will be ignored.
    
\section{Conclusion: Thinking across the fold}

    To think of art as a technology is to think across the fold that separates art from technology, and to begin a process of smoothing out the fold. It challenges us to wonder what it would be like for this difference to not exist, potentially changing the way we think about both artmaking and technological development.
    
    If the difference between art and technology that causes it to be unthinkable to think of art as a technology can be rethought, then perhaps it is a job for artists who work with emerging technologies. 
    
    As artists working at the edges of technological evolution, we are invited to return to an idea of crafting, governed by ethics and informed by an appreciation of the complex, interconnected nature of all things. Perhaps we will open a space in which humans can begin to learn how to craft our technologies differently at the very moment in history when our technologies are learning to relate differently with us.
    
    To paraphrase Gilbert Simondon, we may perhaps begin to think of ourselves as

    \begin{quote}
        inventors of technical and living objects. We coordinate and organise their mutual relation at the level of machines, between machines. [...] We construct the signification of the exchanges of information between machines. Our rapport with the technical object is a coupling between the living and the non–living. \citep[p.xvi]{SimondonOnThMdOfExstncOfTechnclObjcts1980}
    \end{quote}
    
\bibliographystyle{isea}
\bibliography{isea}

\section{Author Biography}

Kynan Stewart Hughes is a PhD candidate at the University of Technology Sydney. His research is at the conjunction of complexity thinking, art and technology. He is a practising artist and has worked in the software industry for over 20 years. 

\end{document}

%     \subsubsection{Epiphenomenal Objects with Indeterminate Meaning}

%     Jason Hoelscher has pointed out that art objects are epiphenomenal, like rainbows \citep[p.17]{HoelscherArtAsInfrmtn2021}. It is always possible to see how they are the result of some other phenomena. The materials from which they are made are always visible, and yet the artwork exists somehow separately from these materials. An art object is made, in a way, out of its own \emph{artness} that emerges from the combination and arrangement of existing objects and materials \citep[p.2]{HoelscherThPtcsOfPhsSpc2014}. Epiphenomena like rainbows and art objects are compelling. They grab our attention by virtue of their complex, contingent mode of existence \citep[p.18]{HoelscherThPtcsOfPhsSpc2014}. 
    
%     \subsubsection{Difference as Information}

%     The resolution of meaningless disparity into meaningful difference is the process by which all things come into existence. It is everywhere and always a wildly creative event that results in an exponentially more vast field of potential for meaningful action – a huge increase in capacity to affect and be affected. Hoelscher has invited us to consider, for example, the evolution of eyes from, perhaps, light–sensitive areas on an organism's surface. These areas eventually become sensitive enough to allow the creature to perceive a consistent difference in the input signal from different eyes. In the macro–scale equivalent of a quantum shift, a three dimensional universe comes into existence \citep[p.5]{HoelscherArtAsInfrmtn2021}.

%     It has been suggested that information is as fundamental a concept for contemporary physics as energy was for classical physics\footnote{
%         ``[...] every item of the physical world has at bottom [...] an immaterial source and explanation; [...] all things physical are information-theoretic in origin and this is a participatory universe.'' \citep{WheelerInfrmtnPhscsQntm2018}
%     }. Hoelscher has blended Claude Shannon's Information Theory with Gilbert Simondon's idea of information arising from meaningless disparity as emergent, meaningful difference. He defines information as the ``relational operation of difference that intensifies or generates a context'' \citep[p.6]{HoelscherArtAsInfrmtn2021}. In embodying meaning, art objects are tapping into the generative power of the universe itself. They function as ``differentially complex'' \citep[p.74]{HoelscherArtAsInfrmtn2021} objects making information available that is otherwise hidden in plain sight as apparently disordered, entropic chaos. In other words, they organise information about their context, making it available as meaning.

%     \subsubsection{Examples: Robert Morris and Marcel Duchamp}

%     \begin{figure}[h]
%         \includegraphics[width=3.31in]{robert-morris-cubes.png}
%         \caption{Robert Morris, American, born 1931. \emph{Untitled (Battered Cubes)}, 1966. Painted Fibreglass (4 Units). Source: https://www.theextravagant.com/lifestyle/art-culture/when-art-is-not-about-art-but-about-you/.}
%         \label{fig:robert-morris-cubes}
%     \end{figure}

%     For an example, Hoelscher uses the hyper minimalist sculptures of Robert Morris, which are simple grey cube–like forms placed directly on the floor of a gallery (Figure~\ref{fig:robert-morris-cubes}).

%     \begin{quote}
%         By giving the viewer so little to look at, Morris [...] directs the flow of attention away from the object itself and toward the object's relations with the gallery space, with the changes of light and shadow over time, and with the viewer's field of experience as an experience in and of itself. Accordingly, with the severe reduction of the work's surface qualities, internal differentiations, particularity, and variability, the artwork and its local situation fold into one another – a direct ingression of the differential information object into the space of lived experience. \citep[p.78]{HoelscherArtAsInfrmtn2021}
%     \end{quote}

%     \begin{figure}[h]
%         \includegraphics[width=3.31in]{snow-shovel.png}
%         \caption{Marcel Duchamp, \emph{In advance of the broken arm}, 1964 (fourth version of the lost 1915 original). Ready–made, snow shovel. New York, Museum of Modern Art. Source: https://www.artesvelata.it/marcel-duchamp/.}
%         \label{fig:snow-shovel}
%     \end{figure}

%     Another example is a Duchamp readymade \emph{In Advance of the Broken Arm} (a snow shovel) (Figure~\ref{fig:snow-shovel}).

%     \begin{quote}
%         [...] the readymade's enfolding of object and idea activates (and is activated by) a wide–ranging network of differential tensions between the object and its context, and between cultural, historical, and artistic expectations regarding shovels, artworks, and the application of artistic skill [...] \citep[p.187]{HoelscherArtAsInfrmtn2021}
%     \end{quote}

%     This is nothing we haven't heard before with respect to a Duchamp readymade. What is interesting is that the functioning of both the Morris sculpture and the readymade, their embodying of meaning, is described in terms of existing information reordering around epiphenomenal art objects. The Morris sculpture reorders information around the object's relation to the gallery space, light and shadow, and the viewer's experience. The Duchamp readymade reorders information around the object's relation to the history of art, the history of shovels, and the history of Duchamp's own work. The reordered information is different, but the process is the same.

%     \subsubsection{Enabling Constraints}

%     Alicia Juarrero's concept of \emph{enabling constraints} is useful as a way to think about meaning as context because it gives artists something to work with.
    
%     Juarrero has identified two types of constraints that define the probability space of complex systems: context–free constraints and context–sensitive constraints. In terms of information, both types of constraints are necessary for reducing randomness in order to separate information from noise but, she says, only context–sensitive constraints enable signals that convey meaning. While both types of constraints function by limiting possibilities, context–sensitive constraints also create possibility by connecting a message/information system that is initially defined by context–free constraints with the complex systems in which it is embedded.
    
%     Context–sensitive constraints create new dimensions of possibility by connecting the system with the state–spaces of other systems. Like coarse–graining, context–sensitive constraints are a way to explain emergence \citep[p.193]{JuarreroThSlfOrgnstnOfIntntnlActn2004} \citep[p.240]{JuarreroCsltyAsCnstrnt1998}.

%     For example, a lighthouse is physically constrained to having only two possible states – it can only blink on and off. Because it is seen within the network of contextual relationships that include navigation, a sailor can understand the flashes of light as meaning "Land!". Signals subject to contextual constraints refer to the contextual web (temporal and spatial) in which that particular signal or event is embedded. The information such signals or processes carry is of the organisation of the network \citep[p.237]{JuarreroCsltyAsCnstrnt1998}
    
%     If we want to add more meaning to the flashing signal, we impose more constraints on it. For example, Morse code puts strict constraints on a simple on/off signal like a flashing light – every combination of flashes must amount to one of the 54 symbols in the code – but the signal information is suddenly of the organisation of a much larger network of potential meaning – of language.

%     The concept of enabling constraints gives us something to grab onto and manipulate. The meaning of an art object emerges from the constraints that connect it with a broader network of things and events. The meaning of an art object is the information it carries about the organisation of this network. This leads to an interesting heuristic for artists (the articulation of a control system perhaps): if we want to add more meaning to an art object, we need to impose more constraints on it.

%     \subsubsection{Example: \emph{Algorithm}}

%     \begin{figure}[h]
%         \includegraphics[width=3.31in]{bubble-sort.png}
%         \caption{Some instances of the \emph{Algorithm} series, 2020. \copyright Respect Copyright.}
%         \label{fig:bubble-sort}
%     \end{figure}

%     A simple example is an experiment of my own: a series of augmented reality sculptures called \emph{Algorithm} (Figure~\ref{fig:bubble-sort}), which rely on a limited set of constraints for their meaning.
    
%     To start, some simple constraints define a limited possibility space that gives the pieces in the series a minimalist stylistic consistency. Each piece in the series is a single wall of greyscale blocks. The number of rows and columns in the wall is variable, as are the dimensions of the blocks, but they all look recognisably similar. The constraints that produce stylistic consistency are not especially meaningful, but they couldn't be described as ``context–free'' because they locate the work within a certain tradition of stylistic minimalism\footnote{
%         It is interesting to consider the possibility that Juarrero's ``context free'' constraints are simply constraints that are not currently meaningful. A stylistic constraint, for example, if the viewer is not familiar with the style, would be effectively context–free.
%     }.
    
%     Each piece in the series depicts a different sorting algorithm from the early history of computing. The blocks animate, sorting themselves from lightest at the bottom to darkest at the top, then randomly scatter, and the sorting process starts again. The blocks move according to the the algorithm they represent. The particular sorting algorithm functions as a meaningful, or ``enabling'' constraint, because the qualitative difference between the pieces is entirely derived from the way the blocks move. Each algorithm is experienced as having a markedly different quality, and it is possible to recognise the particular algorithm from watching the way the blocks move. This series explores the resonance these primitive algorithms have in an era in which their contemporary descendants organise large portions of our lives. Their simple, human–made quality and human scale reference the craft that lies behind information technology.

%   \section{Purpose}

    % The aesthetic regime enables objects created for purposes other than aesthetic appreciation to be experienced as art. As Marcel Duchamp proved in 1917 when he entered a urinal into an art exhibition, and as has been proven many times since by practitioners of the style known as Contemporary Art for whom the readymade is a key strategy, there is nothing in the world that cannot be made into a work of art by the simple act of declaring it to be one. A transfiguration, or dislocation, occurs when an object is brought into the aesthetic regime. The object is no longer understood in terms of its original function or purpose, but as an object of contemplation within the aesthetic regime. ``The aesthetic regime'', as Rancière put it, ``asserts the absolute singularity of art and, at the same time, destroys any pragmatic criterion for isolating this singularity.'' Literally anything can be art.
    
    % Over the last 80 years, the concept of complexity has evolved into a meta-theory, which posits that almost everything can be viewed as a complex system comprised of nested complex systems, each interacting within and across blurred, overlapping boundaries with other systems. Everywhere we look, we see things interacting in ways that would have been impossible to predict, and when we look more closely, we find that the things themselves are emergent approximations of interactions between fleetingly stable regularities in a shifting and chaotic flow. Nothing exists in isolation; everything is interconnected, transient, and contingent. All is process.

    % Over the last several decades, the development of scientific theories of complexity, including Chaos Theory, Dynamical Systems Theory, and Complex Adaptive Systems Theory, constitute a paradigm shift away from simplistic, clockwork models and the search for universal laws, and towards a view that embraces unpredictability, interdependence, and emergence \citep{StengersOrdrOtOfChs1984}. The ancient Western tradition of process philosophy \citep{SeibtStnfrdEncyclpdPrcssPhlsphy1974} has been invigorated by this shift, contextualising knowledge from science and mathematics within a sophisticated materialist ontology.
    
    % This paper draws on both scientific theories of complexity and theories of art and of technology influenced by process philosophy. It is an experiment conducted at the boundary between art and technology which has implications, especially, for artists who work with emerging technologies.
    
    % ``Art as technology'' is proposed as a kind of heuristic for creative technologists of all kinds, opening a space of possibility for crafting differently with art and with technology.

    % \subsubsection{Coarse Graining}
    
    % We can deepen Arthur's definition of technology by applying Jessica Flack's concept of \emph{coarse graining} in place of ``phenomena''. According to Flack, coarse graining occurs when a subsystem with apparently emergent properties is treated as a single entity by components of the system for predictive purposes. Coarse graining is a way for components of complex systems (like people) to treat other elements the system that works but ignores many of the details. It is a ``lossy but true'' \citep[p.4]{FlackCrsGrnng2017} strategy.
    
    % All apparent phenomena are the result of coarse graining. Coarse graining is ``how adaptive systems identify regularities in evolutionary or learning time and use these perceived regularities to guide behaviour." \citep[p.2]{FlackCrsGrnng2017}. It is happening all the time in every system, from coral reef formation \citep[p.61]{FlackEtAlTmsclsSymmtryUncrtnty2013} to the way monkeys establish social hierarchy \citep{FlackCntxtMdltsSgnlMnng2007}.

    % Coarse–graining describes how emergence happens – or at least something about its necessary conditions. In science, a property like temperature emerges, through processes that include coarse–graining, from interactions between physical substances, people measuring, devices, systems of measurement, and the environment \citep[p.4]{FlackCrsGrnng2017}. A distribution of relative status ‘scores’ in a monkey group emerges from a system of signalling interactions between individual monkeys and the perception of those signals by the group \citep{FlackCntxtMdltsSgnlMnng2007}. Undulations in the ocean floor transform into coral reefs as particles collect and aggregate, generating favourable conditions on which complex structures can form \citep[p.61]{FlackEtAlTmsclsSymmtryUncrtnty2013}.
    
    % We might say then, that a technical object is a programming of \emph{perceived regularity} for a purpose.

    % For Kant, the range of aesthetic experience as it pertains to artworks was limited to the experience of beauty. These days, an exclusive focus on beauty when it comes to aesthetic experience and art, seems like an unnecessary limitation to put on aesthetic experience, and on art \citep[pp.121–122]{HighmoreBttrAftrTst2010}. The recent move to equate aesthetic experience with affect is a shift away from the idea that normative experiences of beauty are better than other experiences with artworks, and that artworks are moral lessons, and towards the messy reality of material processes. It contextualises aesthetic experience in a complex ecology of affect. Art is increasingly about ``complex affective and intensive exchanges, situated in the broader ecology of the world.'' \citep[p.155]{HighmoreBttrAftrTst2010}

    % Kant's radical break was to propose that the essence of an art object lies in its capacity to be experienced without concept. With the clarity of hindsight, we can see as Rancière did, that it is the separation of sense–as–in–sensation from sense–as–in–making–sense which created our current situation in which the effectiveness of an artwork, the very thing that makes an object art, is ``a paradoxical kind of efficacy that is produced by the very rupturing of any determinate link between cause and effect.'' Artworks need not make sense. ``It is precisely this \emph{indeterminacy}\footnote{
    %     My emphasis.
    % }'', he said, ``that Kant conceptualized when he defined the beautiful as ‘what is represented as an object of universal delight apart from any concept’''. \citep[p.51-52]{RancierThEmncptdSpcttr2009}
